
% Default to the notebook output style

    


% Inherit from the specified cell style.




    
\documentclass[11pt]{article}

    
    
    \usepackage[T1]{fontenc}
    % Nicer default font (+ math font) than Computer Modern for most use cases
    \usepackage{mathpazo}

    % Basic figure setup, for now with no caption control since it's done
    % automatically by Pandoc (which extracts ![](path) syntax from Markdown).
    \usepackage{graphicx}
    % We will generate all images so they have a width \maxwidth. This means
    % that they will get their normal width if they fit onto the page, but
    % are scaled down if they would overflow the margins.
    \makeatletter
    \def\maxwidth{\ifdim\Gin@nat@width>\linewidth\linewidth
    \else\Gin@nat@width\fi}
    \makeatother
    \let\Oldincludegraphics\includegraphics
    % Set max figure width to be 80% of text width, for now hardcoded.
    \renewcommand{\includegraphics}[1]{\Oldincludegraphics[width=.8\maxwidth]{#1}}
    % Ensure that by default, figures have no caption (until we provide a
    % proper Figure object with a Caption API and a way to capture that
    % in the conversion process - todo).
    \usepackage{caption}
    \DeclareCaptionLabelFormat{nolabel}{}
    \captionsetup{labelformat=nolabel}

    \usepackage{adjustbox} % Used to constrain images to a maximum size 
    \usepackage{xcolor} % Allow colors to be defined
    \usepackage{enumerate} % Needed for markdown enumerations to work
    \usepackage{geometry} % Used to adjust the document margins
    \usepackage{amsmath} % Equations
    \usepackage{amssymb} % Equations
    \usepackage{textcomp} % defines textquotesingle
    % Hack from http://tex.stackexchange.com/a/47451/13684:
    \AtBeginDocument{%
        \def\PYZsq{\textquotesingle}% Upright quotes in Pygmentized code
    }
    \usepackage{upquote} % Upright quotes for verbatim code
    \usepackage{eurosym} % defines \euro
    \usepackage[mathletters]{ucs} % Extended unicode (utf-8) support
    \usepackage[utf8x]{inputenc} % Allow utf-8 characters in the tex document
    \usepackage{fancyvrb} % verbatim replacement that allows latex
    \usepackage{grffile} % extends the file name processing of package graphics 
                         % to support a larger range 
    % The hyperref package gives us a pdf with properly built
    % internal navigation ('pdf bookmarks' for the table of contents,
    % internal cross-reference links, web links for URLs, etc.)
    \usepackage{hyperref}
    \usepackage{longtable} % longtable support required by pandoc >1.10
    \usepackage{booktabs}  % table support for pandoc > 1.12.2
    \usepackage[inline]{enumitem} % IRkernel/repr support (it uses the enumerate* environment)
    \usepackage[normalem]{ulem} % ulem is needed to support strikethroughs (\sout)
                                % normalem makes italics be italics, not underlines
    

    
    
    % Colors for the hyperref package
    \definecolor{urlcolor}{rgb}{0,.145,.698}
    \definecolor{linkcolor}{rgb}{.71,0.21,0.01}
    \definecolor{citecolor}{rgb}{.12,.54,.11}

    % ANSI colors
    \definecolor{ansi-black}{HTML}{3E424D}
    \definecolor{ansi-black-intense}{HTML}{282C36}
    \definecolor{ansi-red}{HTML}{E75C58}
    \definecolor{ansi-red-intense}{HTML}{B22B31}
    \definecolor{ansi-green}{HTML}{00A250}
    \definecolor{ansi-green-intense}{HTML}{007427}
    \definecolor{ansi-yellow}{HTML}{DDB62B}
    \definecolor{ansi-yellow-intense}{HTML}{B27D12}
    \definecolor{ansi-blue}{HTML}{208FFB}
    \definecolor{ansi-blue-intense}{HTML}{0065CA}
    \definecolor{ansi-magenta}{HTML}{D160C4}
    \definecolor{ansi-magenta-intense}{HTML}{A03196}
    \definecolor{ansi-cyan}{HTML}{60C6C8}
    \definecolor{ansi-cyan-intense}{HTML}{258F8F}
    \definecolor{ansi-white}{HTML}{C5C1B4}
    \definecolor{ansi-white-intense}{HTML}{A1A6B2}

    % commands and environments needed by pandoc snippets
    % extracted from the output of `pandoc -s`
    \providecommand{\tightlist}{%
      \setlength{\itemsep}{0pt}\setlength{\parskip}{0pt}}
    \DefineVerbatimEnvironment{Highlighting}{Verbatim}{commandchars=\\\{\}}
    % Add ',fontsize=\small' for more characters per line
    \newenvironment{Shaded}{}{}
    \newcommand{\KeywordTok}[1]{\textcolor[rgb]{0.00,0.44,0.13}{\textbf{{#1}}}}
    \newcommand{\DataTypeTok}[1]{\textcolor[rgb]{0.56,0.13,0.00}{{#1}}}
    \newcommand{\DecValTok}[1]{\textcolor[rgb]{0.25,0.63,0.44}{{#1}}}
    \newcommand{\BaseNTok}[1]{\textcolor[rgb]{0.25,0.63,0.44}{{#1}}}
    \newcommand{\FloatTok}[1]{\textcolor[rgb]{0.25,0.63,0.44}{{#1}}}
    \newcommand{\CharTok}[1]{\textcolor[rgb]{0.25,0.44,0.63}{{#1}}}
    \newcommand{\StringTok}[1]{\textcolor[rgb]{0.25,0.44,0.63}{{#1}}}
    \newcommand{\CommentTok}[1]{\textcolor[rgb]{0.38,0.63,0.69}{\textit{{#1}}}}
    \newcommand{\OtherTok}[1]{\textcolor[rgb]{0.00,0.44,0.13}{{#1}}}
    \newcommand{\AlertTok}[1]{\textcolor[rgb]{1.00,0.00,0.00}{\textbf{{#1}}}}
    \newcommand{\FunctionTok}[1]{\textcolor[rgb]{0.02,0.16,0.49}{{#1}}}
    \newcommand{\RegionMarkerTok}[1]{{#1}}
    \newcommand{\ErrorTok}[1]{\textcolor[rgb]{1.00,0.00,0.00}{\textbf{{#1}}}}
    \newcommand{\NormalTok}[1]{{#1}}
    
    % Additional commands for more recent versions of Pandoc
    \newcommand{\ConstantTok}[1]{\textcolor[rgb]{0.53,0.00,0.00}{{#1}}}
    \newcommand{\SpecialCharTok}[1]{\textcolor[rgb]{0.25,0.44,0.63}{{#1}}}
    \newcommand{\VerbatimStringTok}[1]{\textcolor[rgb]{0.25,0.44,0.63}{{#1}}}
    \newcommand{\SpecialStringTok}[1]{\textcolor[rgb]{0.73,0.40,0.53}{{#1}}}
    \newcommand{\ImportTok}[1]{{#1}}
    \newcommand{\DocumentationTok}[1]{\textcolor[rgb]{0.73,0.13,0.13}{\textit{{#1}}}}
    \newcommand{\AnnotationTok}[1]{\textcolor[rgb]{0.38,0.63,0.69}{\textbf{\textit{{#1}}}}}
    \newcommand{\CommentVarTok}[1]{\textcolor[rgb]{0.38,0.63,0.69}{\textbf{\textit{{#1}}}}}
    \newcommand{\VariableTok}[1]{\textcolor[rgb]{0.10,0.09,0.49}{{#1}}}
    \newcommand{\ControlFlowTok}[1]{\textcolor[rgb]{0.00,0.44,0.13}{\textbf{{#1}}}}
    \newcommand{\OperatorTok}[1]{\textcolor[rgb]{0.40,0.40,0.40}{{#1}}}
    \newcommand{\BuiltInTok}[1]{{#1}}
    \newcommand{\ExtensionTok}[1]{{#1}}
    \newcommand{\PreprocessorTok}[1]{\textcolor[rgb]{0.74,0.48,0.00}{{#1}}}
    \newcommand{\AttributeTok}[1]{\textcolor[rgb]{0.49,0.56,0.16}{{#1}}}
    \newcommand{\InformationTok}[1]{\textcolor[rgb]{0.38,0.63,0.69}{\textbf{\textit{{#1}}}}}
    \newcommand{\WarningTok}[1]{\textcolor[rgb]{0.38,0.63,0.69}{\textbf{\textit{{#1}}}}}
    
    
    % Define a nice break command that doesn't care if a line doesn't already
    % exist.
    \def\br{\hspace*{\fill} \\* }
    % Math Jax compatability definitions
    \def\gt{>}
    \def\lt{<}
    % Document parameters
    \title{kfold\_nn\_hw}
    
    
    

    % Pygments definitions
    
\makeatletter
\def\PY@reset{\let\PY@it=\relax \let\PY@bf=\relax%
    \let\PY@ul=\relax \let\PY@tc=\relax%
    \let\PY@bc=\relax \let\PY@ff=\relax}
\def\PY@tok#1{\csname PY@tok@#1\endcsname}
\def\PY@toks#1+{\ifx\relax#1\empty\else%
    \PY@tok{#1}\expandafter\PY@toks\fi}
\def\PY@do#1{\PY@bc{\PY@tc{\PY@ul{%
    \PY@it{\PY@bf{\PY@ff{#1}}}}}}}
\def\PY#1#2{\PY@reset\PY@toks#1+\relax+\PY@do{#2}}

\expandafter\def\csname PY@tok@w\endcsname{\def\PY@tc##1{\textcolor[rgb]{0.73,0.73,0.73}{##1}}}
\expandafter\def\csname PY@tok@c\endcsname{\let\PY@it=\textit\def\PY@tc##1{\textcolor[rgb]{0.25,0.50,0.50}{##1}}}
\expandafter\def\csname PY@tok@cp\endcsname{\def\PY@tc##1{\textcolor[rgb]{0.74,0.48,0.00}{##1}}}
\expandafter\def\csname PY@tok@k\endcsname{\let\PY@bf=\textbf\def\PY@tc##1{\textcolor[rgb]{0.00,0.50,0.00}{##1}}}
\expandafter\def\csname PY@tok@kp\endcsname{\def\PY@tc##1{\textcolor[rgb]{0.00,0.50,0.00}{##1}}}
\expandafter\def\csname PY@tok@kt\endcsname{\def\PY@tc##1{\textcolor[rgb]{0.69,0.00,0.25}{##1}}}
\expandafter\def\csname PY@tok@o\endcsname{\def\PY@tc##1{\textcolor[rgb]{0.40,0.40,0.40}{##1}}}
\expandafter\def\csname PY@tok@ow\endcsname{\let\PY@bf=\textbf\def\PY@tc##1{\textcolor[rgb]{0.67,0.13,1.00}{##1}}}
\expandafter\def\csname PY@tok@nb\endcsname{\def\PY@tc##1{\textcolor[rgb]{0.00,0.50,0.00}{##1}}}
\expandafter\def\csname PY@tok@nf\endcsname{\def\PY@tc##1{\textcolor[rgb]{0.00,0.00,1.00}{##1}}}
\expandafter\def\csname PY@tok@nc\endcsname{\let\PY@bf=\textbf\def\PY@tc##1{\textcolor[rgb]{0.00,0.00,1.00}{##1}}}
\expandafter\def\csname PY@tok@nn\endcsname{\let\PY@bf=\textbf\def\PY@tc##1{\textcolor[rgb]{0.00,0.00,1.00}{##1}}}
\expandafter\def\csname PY@tok@ne\endcsname{\let\PY@bf=\textbf\def\PY@tc##1{\textcolor[rgb]{0.82,0.25,0.23}{##1}}}
\expandafter\def\csname PY@tok@nv\endcsname{\def\PY@tc##1{\textcolor[rgb]{0.10,0.09,0.49}{##1}}}
\expandafter\def\csname PY@tok@no\endcsname{\def\PY@tc##1{\textcolor[rgb]{0.53,0.00,0.00}{##1}}}
\expandafter\def\csname PY@tok@nl\endcsname{\def\PY@tc##1{\textcolor[rgb]{0.63,0.63,0.00}{##1}}}
\expandafter\def\csname PY@tok@ni\endcsname{\let\PY@bf=\textbf\def\PY@tc##1{\textcolor[rgb]{0.60,0.60,0.60}{##1}}}
\expandafter\def\csname PY@tok@na\endcsname{\def\PY@tc##1{\textcolor[rgb]{0.49,0.56,0.16}{##1}}}
\expandafter\def\csname PY@tok@nt\endcsname{\let\PY@bf=\textbf\def\PY@tc##1{\textcolor[rgb]{0.00,0.50,0.00}{##1}}}
\expandafter\def\csname PY@tok@nd\endcsname{\def\PY@tc##1{\textcolor[rgb]{0.67,0.13,1.00}{##1}}}
\expandafter\def\csname PY@tok@s\endcsname{\def\PY@tc##1{\textcolor[rgb]{0.73,0.13,0.13}{##1}}}
\expandafter\def\csname PY@tok@sd\endcsname{\let\PY@it=\textit\def\PY@tc##1{\textcolor[rgb]{0.73,0.13,0.13}{##1}}}
\expandafter\def\csname PY@tok@si\endcsname{\let\PY@bf=\textbf\def\PY@tc##1{\textcolor[rgb]{0.73,0.40,0.53}{##1}}}
\expandafter\def\csname PY@tok@se\endcsname{\let\PY@bf=\textbf\def\PY@tc##1{\textcolor[rgb]{0.73,0.40,0.13}{##1}}}
\expandafter\def\csname PY@tok@sr\endcsname{\def\PY@tc##1{\textcolor[rgb]{0.73,0.40,0.53}{##1}}}
\expandafter\def\csname PY@tok@ss\endcsname{\def\PY@tc##1{\textcolor[rgb]{0.10,0.09,0.49}{##1}}}
\expandafter\def\csname PY@tok@sx\endcsname{\def\PY@tc##1{\textcolor[rgb]{0.00,0.50,0.00}{##1}}}
\expandafter\def\csname PY@tok@m\endcsname{\def\PY@tc##1{\textcolor[rgb]{0.40,0.40,0.40}{##1}}}
\expandafter\def\csname PY@tok@gh\endcsname{\let\PY@bf=\textbf\def\PY@tc##1{\textcolor[rgb]{0.00,0.00,0.50}{##1}}}
\expandafter\def\csname PY@tok@gu\endcsname{\let\PY@bf=\textbf\def\PY@tc##1{\textcolor[rgb]{0.50,0.00,0.50}{##1}}}
\expandafter\def\csname PY@tok@gd\endcsname{\def\PY@tc##1{\textcolor[rgb]{0.63,0.00,0.00}{##1}}}
\expandafter\def\csname PY@tok@gi\endcsname{\def\PY@tc##1{\textcolor[rgb]{0.00,0.63,0.00}{##1}}}
\expandafter\def\csname PY@tok@gr\endcsname{\def\PY@tc##1{\textcolor[rgb]{1.00,0.00,0.00}{##1}}}
\expandafter\def\csname PY@tok@ge\endcsname{\let\PY@it=\textit}
\expandafter\def\csname PY@tok@gs\endcsname{\let\PY@bf=\textbf}
\expandafter\def\csname PY@tok@gp\endcsname{\let\PY@bf=\textbf\def\PY@tc##1{\textcolor[rgb]{0.00,0.00,0.50}{##1}}}
\expandafter\def\csname PY@tok@go\endcsname{\def\PY@tc##1{\textcolor[rgb]{0.53,0.53,0.53}{##1}}}
\expandafter\def\csname PY@tok@gt\endcsname{\def\PY@tc##1{\textcolor[rgb]{0.00,0.27,0.87}{##1}}}
\expandafter\def\csname PY@tok@err\endcsname{\def\PY@bc##1{\setlength{\fboxsep}{0pt}\fcolorbox[rgb]{1.00,0.00,0.00}{1,1,1}{\strut ##1}}}
\expandafter\def\csname PY@tok@kc\endcsname{\let\PY@bf=\textbf\def\PY@tc##1{\textcolor[rgb]{0.00,0.50,0.00}{##1}}}
\expandafter\def\csname PY@tok@kd\endcsname{\let\PY@bf=\textbf\def\PY@tc##1{\textcolor[rgb]{0.00,0.50,0.00}{##1}}}
\expandafter\def\csname PY@tok@kn\endcsname{\let\PY@bf=\textbf\def\PY@tc##1{\textcolor[rgb]{0.00,0.50,0.00}{##1}}}
\expandafter\def\csname PY@tok@kr\endcsname{\let\PY@bf=\textbf\def\PY@tc##1{\textcolor[rgb]{0.00,0.50,0.00}{##1}}}
\expandafter\def\csname PY@tok@bp\endcsname{\def\PY@tc##1{\textcolor[rgb]{0.00,0.50,0.00}{##1}}}
\expandafter\def\csname PY@tok@fm\endcsname{\def\PY@tc##1{\textcolor[rgb]{0.00,0.00,1.00}{##1}}}
\expandafter\def\csname PY@tok@vc\endcsname{\def\PY@tc##1{\textcolor[rgb]{0.10,0.09,0.49}{##1}}}
\expandafter\def\csname PY@tok@vg\endcsname{\def\PY@tc##1{\textcolor[rgb]{0.10,0.09,0.49}{##1}}}
\expandafter\def\csname PY@tok@vi\endcsname{\def\PY@tc##1{\textcolor[rgb]{0.10,0.09,0.49}{##1}}}
\expandafter\def\csname PY@tok@vm\endcsname{\def\PY@tc##1{\textcolor[rgb]{0.10,0.09,0.49}{##1}}}
\expandafter\def\csname PY@tok@sa\endcsname{\def\PY@tc##1{\textcolor[rgb]{0.73,0.13,0.13}{##1}}}
\expandafter\def\csname PY@tok@sb\endcsname{\def\PY@tc##1{\textcolor[rgb]{0.73,0.13,0.13}{##1}}}
\expandafter\def\csname PY@tok@sc\endcsname{\def\PY@tc##1{\textcolor[rgb]{0.73,0.13,0.13}{##1}}}
\expandafter\def\csname PY@tok@dl\endcsname{\def\PY@tc##1{\textcolor[rgb]{0.73,0.13,0.13}{##1}}}
\expandafter\def\csname PY@tok@s2\endcsname{\def\PY@tc##1{\textcolor[rgb]{0.73,0.13,0.13}{##1}}}
\expandafter\def\csname PY@tok@sh\endcsname{\def\PY@tc##1{\textcolor[rgb]{0.73,0.13,0.13}{##1}}}
\expandafter\def\csname PY@tok@s1\endcsname{\def\PY@tc##1{\textcolor[rgb]{0.73,0.13,0.13}{##1}}}
\expandafter\def\csname PY@tok@mb\endcsname{\def\PY@tc##1{\textcolor[rgb]{0.40,0.40,0.40}{##1}}}
\expandafter\def\csname PY@tok@mf\endcsname{\def\PY@tc##1{\textcolor[rgb]{0.40,0.40,0.40}{##1}}}
\expandafter\def\csname PY@tok@mh\endcsname{\def\PY@tc##1{\textcolor[rgb]{0.40,0.40,0.40}{##1}}}
\expandafter\def\csname PY@tok@mi\endcsname{\def\PY@tc##1{\textcolor[rgb]{0.40,0.40,0.40}{##1}}}
\expandafter\def\csname PY@tok@il\endcsname{\def\PY@tc##1{\textcolor[rgb]{0.40,0.40,0.40}{##1}}}
\expandafter\def\csname PY@tok@mo\endcsname{\def\PY@tc##1{\textcolor[rgb]{0.40,0.40,0.40}{##1}}}
\expandafter\def\csname PY@tok@ch\endcsname{\let\PY@it=\textit\def\PY@tc##1{\textcolor[rgb]{0.25,0.50,0.50}{##1}}}
\expandafter\def\csname PY@tok@cm\endcsname{\let\PY@it=\textit\def\PY@tc##1{\textcolor[rgb]{0.25,0.50,0.50}{##1}}}
\expandafter\def\csname PY@tok@cpf\endcsname{\let\PY@it=\textit\def\PY@tc##1{\textcolor[rgb]{0.25,0.50,0.50}{##1}}}
\expandafter\def\csname PY@tok@c1\endcsname{\let\PY@it=\textit\def\PY@tc##1{\textcolor[rgb]{0.25,0.50,0.50}{##1}}}
\expandafter\def\csname PY@tok@cs\endcsname{\let\PY@it=\textit\def\PY@tc##1{\textcolor[rgb]{0.25,0.50,0.50}{##1}}}

\def\PYZbs{\char`\\}
\def\PYZus{\char`\_}
\def\PYZob{\char`\{}
\def\PYZcb{\char`\}}
\def\PYZca{\char`\^}
\def\PYZam{\char`\&}
\def\PYZlt{\char`\<}
\def\PYZgt{\char`\>}
\def\PYZsh{\char`\#}
\def\PYZpc{\char`\%}
\def\PYZdl{\char`\$}
\def\PYZhy{\char`\-}
\def\PYZsq{\char`\'}
\def\PYZdq{\char`\"}
\def\PYZti{\char`\~}
% for compatibility with earlier versions
\def\PYZat{@}
\def\PYZlb{[}
\def\PYZrb{]}
\makeatother


    % Exact colors from NB
    \definecolor{incolor}{rgb}{0.0, 0.0, 0.5}
    \definecolor{outcolor}{rgb}{0.545, 0.0, 0.0}



    
    % Prevent overflowing lines due to hard-to-break entities
    \sloppy 
    % Setup hyperref package
    \hypersetup{
      breaklinks=true,  % so long urls are correctly broken across lines
      colorlinks=true,
      urlcolor=urlcolor,
      linkcolor=linkcolor,
      citecolor=citecolor,
      }
    % Slightly bigger margins than the latex defaults
    
    \geometry{verbose,tmargin=1in,bmargin=1in,lmargin=1in,rmargin=1in}
    
    

    \begin{document}
    
    
    \maketitle
    
    

    
    \section{KFolds and NN Homework}\label{kfolds-and-nn-homework}

Data Mining 2\\
20180912\\
Ron Neely

Redo the breast cancer example in R or Python using 5-Fold Cross
Validation. Remember to use each sample only once.

\begin{itemize}
\tightlist
\item
  Read data in to a pandas DataFrame and normalize (python)
\item
  Establish a baseline model using sklearn MLPClassifier from class and
  show metrics
\item
  Repeat with kfolds split and test and compare metrics
\end{itemize}

Remember a competent answer to 1. is worth 9/10. Ask and answer a
challenge question on the topic of Neural Networks or K-Folds for the
10th point. (Doesn't have to be hard. Just has to be something not
covered in class.)

\textbf{Challenge question}: Investigate whether "m-folds validation"
variation of "k-folds" be used as a basis to identify and remove
outliers for an improved model. Prediction: "m-folds" will either be
novel idea or about as dumb as a square wheel.

"m-folds validation" will

\begin{itemize}
\tightlist
\item
  for every m row:

  \begin{itemize}
  \tightlist
  \item
    train on all the other rows
  \item
    predict the m\_th row with that model
  \item
    keep the m\_th row if prediction was accurate
  \item
    discard the m\_th row if prediction was inaccurate, i.e. remove it
    as an outlier
  \end{itemize}
\item
  train a model all the kept rows (i.e. without removed outliers)
\item
  evaluate and compare metrics with previously predicted models
\end{itemize}

    \begin{Verbatim}[commandchars=\\\{\}]
{\color{incolor}In [{\color{incolor}1}]:} \PY{o}{\PYZpc{}}\PY{k}{matplotlib} inline
        \PY{k+kn}{import} \PY{n+nn}{matplotlib}\PY{n+nn}{.}\PY{n+nn}{pyplot} \PY{k}{as} \PY{n+nn}{plt}
        \PY{k+kn}{import} \PY{n+nn}{pandas} \PY{k}{as} \PY{n+nn}{pd}
        \PY{k+kn}{import} \PY{n+nn}{numpy} \PY{k}{as} \PY{n+nn}{np}
        \PY{k+kn}{from} \PY{n+nn}{sklearn} \PY{k}{import} \PY{n}{datasets}
        \PY{k+kn}{from} \PY{n+nn}{sklearn}\PY{n+nn}{.}\PY{n+nn}{preprocessing} \PY{k}{import} \PY{n}{StandardScaler}
        \PY{k+kn}{from} \PY{n+nn}{sklearn}\PY{n+nn}{.}\PY{n+nn}{model\PYZus{}selection} \PY{k}{import} \PY{n}{train\PYZus{}test\PYZus{}split}
        \PY{k+kn}{from} \PY{n+nn}{sklearn}\PY{n+nn}{.}\PY{n+nn}{neural\PYZus{}network} \PY{k}{import} \PY{n}{MLPClassifier}
        \PY{k+kn}{from} \PY{n+nn}{sklearn}\PY{n+nn}{.}\PY{n+nn}{metrics} \PY{k}{import} \PY{n}{classification\PYZus{}report}\PY{p}{,}\PY{n}{confusion\PYZus{}matrix}
        \PY{k+kn}{from} \PY{n+nn}{sklearn}\PY{n+nn}{.}\PY{n+nn}{model\PYZus{}selection} \PY{k}{import} \PY{n}{KFold}
        \PY{k+kn}{import} \PY{n+nn}{sys}
        \PY{n}{sys}\PY{o}{.}\PY{n}{version}
\end{Verbatim}


\begin{Verbatim}[commandchars=\\\{\}]
{\color{outcolor}Out[{\color{outcolor}1}]:} '3.6.6 |Anaconda custom (64-bit)| (default, Jun 28 2018, 11:27:44) [MSC v.1900 64 bit (AMD64)]'
\end{Verbatim}
            
    \subsection{Read data in to a pandas DataFrame and
normalize}\label{read-data-in-to-a-pandas-dataframe-and-normalize}

    \begin{Verbatim}[commandchars=\\\{\}]
{\color{incolor}In [{\color{incolor}2}]:} \PY{k}{def} \PY{n+nf}{sklearn\PYZus{}to\PYZus{}df}\PY{p}{(}\PY{n}{sklearn\PYZus{}dataset}\PY{p}{)}\PY{p}{:}
            \PY{n}{df} \PY{o}{=} \PY{n}{pd}\PY{o}{.}\PY{n}{DataFrame}\PY{p}{(}\PY{n}{sklearn\PYZus{}dataset}\PY{o}{.}\PY{n}{data}\PY{p}{,} \PY{n}{columns}\PY{o}{=}\PY{n}{sklearn\PYZus{}dataset}\PY{o}{.}\PY{n}{feature\PYZus{}names}\PY{p}{)}
            \PY{n}{df}\PY{p}{[}\PY{l+s+s1}{\PYZsq{}}\PY{l+s+s1}{y}\PY{l+s+s1}{\PYZsq{}}\PY{p}{]} \PY{o}{=} \PY{n}{pd}\PY{o}{.}\PY{n}{Series}\PY{p}{(}\PY{n}{sklearn\PYZus{}dataset}\PY{o}{.}\PY{n}{target}\PY{p}{)}
            \PY{k}{return} \PY{n}{df}
        
        \PY{n}{df} \PY{o}{=} \PY{n}{sklearn\PYZus{}to\PYZus{}df}\PY{p}{(}\PY{n}{datasets}\PY{o}{.}\PY{n}{load\PYZus{}breast\PYZus{}cancer}\PY{p}{(}\PY{p}{)}\PY{p}{)}
        \PY{n+nb}{print}\PY{p}{(}\PY{n}{df}\PY{o}{.}\PY{n}{shape}\PY{p}{)}
        \PY{n+nb}{print}\PY{p}{(}\PY{n}{df}\PY{o}{.}\PY{n}{dtypes}\PY{p}{,} \PY{n}{end}\PY{o}{=}\PY{l+s+s1}{\PYZsq{}}\PY{l+s+s1}{ }\PY{l+s+s1}{\PYZsq{}}\PY{p}{)}
        \PY{n}{df}\PY{o}{.}\PY{n}{head}\PY{p}{(}\PY{l+m+mi}{3}\PY{p}{)}
\end{Verbatim}


    \begin{Verbatim}[commandchars=\\\{\}]
(569, 31)
mean radius                float64
mean texture               float64
mean perimeter             float64
mean area                  float64
mean smoothness            float64
mean compactness           float64
mean concavity             float64
mean concave points        float64
mean symmetry              float64
mean fractal dimension     float64
radius error               float64
texture error              float64
perimeter error            float64
area error                 float64
smoothness error           float64
compactness error          float64
concavity error            float64
concave points error       float64
symmetry error             float64
fractal dimension error    float64
worst radius               float64
worst texture              float64
worst perimeter            float64
worst area                 float64
worst smoothness           float64
worst compactness          float64
worst concavity            float64
worst concave points       float64
worst symmetry             float64
worst fractal dimension    float64
y                            int32
dtype: object 
    \end{Verbatim}

\begin{Verbatim}[commandchars=\\\{\}]
{\color{outcolor}Out[{\color{outcolor}2}]:}    mean radius  mean texture  mean perimeter  mean area  mean smoothness  \textbackslash{}
        0        17.99         10.38           122.8     1001.0          0.11840   
        1        20.57         17.77           132.9     1326.0          0.08474   
        2        19.69         21.25           130.0     1203.0          0.10960   
        
           mean compactness  mean concavity  mean concave points  mean symmetry  \textbackslash{}
        0           0.27760          0.3001              0.14710         0.2419   
        1           0.07864          0.0869              0.07017         0.1812   
        2           0.15990          0.1974              0.12790         0.2069   
        
           mean fractal dimension {\ldots}  worst texture  worst perimeter  worst area  \textbackslash{}
        0                 0.07871 {\ldots}          17.33            184.6      2019.0   
        1                 0.05667 {\ldots}          23.41            158.8      1956.0   
        2                 0.05999 {\ldots}          25.53            152.5      1709.0   
        
           worst smoothness  worst compactness  worst concavity  worst concave points  \textbackslash{}
        0            0.1622             0.6656           0.7119                0.2654   
        1            0.1238             0.1866           0.2416                0.1860   
        2            0.1444             0.4245           0.4504                0.2430   
        
           worst symmetry  worst fractal dimension  y  
        0          0.4601                  0.11890  0  
        1          0.2750                  0.08902  0  
        2          0.3613                  0.08758  0  
        
        [3 rows x 31 columns]
\end{Verbatim}
            
    \begin{Verbatim}[commandchars=\\\{\}]
{\color{incolor}In [{\color{incolor}3}]:} \PY{n}{y} \PY{o}{=} \PY{n}{df}\PY{p}{[}\PY{l+s+s1}{\PYZsq{}}\PY{l+s+s1}{y}\PY{l+s+s1}{\PYZsq{}}\PY{p}{]}
        \PY{n+nb}{print}\PY{p}{(}\PY{n+nb}{len}\PY{p}{(}\PY{n}{y}\PY{p}{)}\PY{p}{)}
        \PY{n+nb}{print}\PY{p}{(}\PY{n}{y}\PY{o}{.}\PY{n}{dtype}\PY{p}{)}
        \PY{n}{y}\PY{o}{.}\PY{n}{head}\PY{p}{(}\PY{l+m+mi}{3}\PY{p}{)}
\end{Verbatim}


    \begin{Verbatim}[commandchars=\\\{\}]
569
int32

    \end{Verbatim}

\begin{Verbatim}[commandchars=\\\{\}]
{\color{outcolor}Out[{\color{outcolor}3}]:} 0    0
        1    0
        2    0
        Name: y, dtype: int32
\end{Verbatim}
            
    \begin{Verbatim}[commandchars=\\\{\}]
{\color{incolor}In [{\color{incolor}4}]:} \PY{n}{X\PYZus{}cols} \PY{o}{=} \PY{n}{df}\PY{o}{.}\PY{n}{columns}\PY{p}{[}\PY{l+m+mi}{0}\PY{p}{:}\PY{o}{\PYZhy{}}\PY{l+m+mi}{1}\PY{p}{]}
        \PY{n}{scaler} \PY{o}{=} \PY{n}{StandardScaler}\PY{p}{(}\PY{p}{)}
        \PY{n}{df}\PY{p}{[}\PY{n}{X\PYZus{}cols}\PY{p}{]} \PY{o}{=} \PY{n}{scaler}\PY{o}{.}\PY{n}{fit\PYZus{}transform}\PY{p}{(}\PY{n}{df}\PY{p}{[}\PY{n}{X\PYZus{}cols}\PY{p}{]}\PY{p}{)}
        \PY{n}{X} \PY{o}{=} \PY{n}{df}\PY{p}{[}\PY{n}{X\PYZus{}cols}\PY{p}{]}
        \PY{n+nb}{print}\PY{p}{(}\PY{n}{X}\PY{o}{.}\PY{n}{shape}\PY{p}{)}
        \PY{n}{X}\PY{o}{.}\PY{n}{head}\PY{p}{(}\PY{l+m+mi}{3}\PY{p}{)}
\end{Verbatim}


    \begin{Verbatim}[commandchars=\\\{\}]
(569, 30)

    \end{Verbatim}

\begin{Verbatim}[commandchars=\\\{\}]
{\color{outcolor}Out[{\color{outcolor}4}]:}    mean radius  mean texture  mean perimeter  mean area  mean smoothness  \textbackslash{}
        0     1.097064     -2.073335        1.269934   0.984375         1.568466   
        1     1.829821     -0.353632        1.685955   1.908708        -0.826962   
        2     1.579888      0.456187        1.566503   1.558884         0.942210   
        
           mean compactness  mean concavity  mean concave points  mean symmetry  \textbackslash{}
        0          3.283515        2.652874             2.532475       2.217515   
        1         -0.487072       -0.023846             0.548144       0.001392   
        2          1.052926        1.363478             2.037231       0.939685   
        
           mean fractal dimension           {\ldots}             worst radius  \textbackslash{}
        0                2.255747           {\ldots}                 1.886690   
        1               -0.868652           {\ldots}                 1.805927   
        2               -0.398008           {\ldots}                 1.511870   
        
           worst texture  worst perimeter  worst area  worst smoothness  \textbackslash{}
        0      -1.359293         2.303601    2.001237          1.307686   
        1      -0.369203         1.535126    1.890489         -0.375612   
        2      -0.023974         1.347475    1.456285          0.527407   
        
           worst compactness  worst concavity  worst concave points  worst symmetry  \textbackslash{}
        0           2.616665         2.109526              2.296076        2.750622   
        1          -0.430444        -0.146749              1.087084       -0.243890   
        2           1.082932         0.854974              1.955000        1.152255   
        
           worst fractal dimension  
        0                 1.937015  
        1                 0.281190  
        2                 0.201391  
        
        [3 rows x 30 columns]
\end{Verbatim}
            
    \subsection{Establish a baseline model using sklearn MLPClassifier from
class and show
metrics}\label{establish-a-baseline-model-using-sklearn-mlpclassifier-from-class-and-show-metrics}

    \begin{Verbatim}[commandchars=\\\{\}]
{\color{incolor}In [{\color{incolor}5}]:} \PY{n}{X\PYZus{}train}\PY{p}{,} \PY{n}{X\PYZus{}test}\PY{p}{,} \PY{n}{y\PYZus{}train}\PY{p}{,} \PY{n}{y\PYZus{}test} \PY{o}{=} \PY{n}{train\PYZus{}test\PYZus{}split}\PY{p}{(}\PY{n}{X}\PY{p}{,} \PY{n}{y}\PY{p}{)}
        \PY{n+nb}{print}\PY{p}{(}\PY{n}{X\PYZus{}train}\PY{o}{.}\PY{n}{shape}\PY{p}{,} \PY{n}{X\PYZus{}test}\PY{o}{.}\PY{n}{shape}\PY{p}{,} \PY{n}{y\PYZus{}train}\PY{o}{.}\PY{n}{shape}\PY{p}{,} \PY{n}{y\PYZus{}test}\PY{o}{.}\PY{n}{shape}\PY{p}{)}
\end{Verbatim}


    \begin{Verbatim}[commandchars=\\\{\}]
(426, 30) (143, 30) (426,) (143,)

    \end{Verbatim}

    Looks like default split was about 75\% train, 25\% test.

    \begin{Verbatim}[commandchars=\\\{\}]
{\color{incolor}In [{\color{incolor}6}]:} \PY{n}{mlp} \PY{o}{=} \PY{n}{MLPClassifier}\PY{p}{(}\PY{n}{hidden\PYZus{}layer\PYZus{}sizes}\PY{o}{=}\PY{p}{(}\PY{l+m+mi}{30}\PY{p}{,}\PY{l+m+mi}{30}\PY{p}{,}\PY{l+m+mi}{30}\PY{p}{)}\PY{p}{)}
        \PY{n}{mlp}\PY{o}{.}\PY{n}{fit}\PY{p}{(}\PY{n}{X\PYZus{}train}\PY{p}{,}\PY{n}{y\PYZus{}train}\PY{p}{)}
\end{Verbatim}


\begin{Verbatim}[commandchars=\\\{\}]
{\color{outcolor}Out[{\color{outcolor}6}]:} MLPClassifier(activation='relu', alpha=0.0001, batch\_size='auto', beta\_1=0.9,
               beta\_2=0.999, early\_stopping=False, epsilon=1e-08,
               hidden\_layer\_sizes=(30, 30, 30), learning\_rate='constant',
               learning\_rate\_init=0.001, max\_iter=200, momentum=0.9,
               nesterovs\_momentum=True, power\_t=0.5, random\_state=None,
               shuffle=True, solver='adam', tol=0.0001, validation\_fraction=0.1,
               verbose=False, warm\_start=False)
\end{Verbatim}
            
    \subsection{Metrics for basic model (no
k-folds)}\label{metrics-for-basic-model-no-k-folds}

    \begin{Verbatim}[commandchars=\\\{\}]
{\color{incolor}In [{\color{incolor}7}]:} \PY{n}{y\PYZus{}pred} \PY{o}{=} \PY{n}{mlp}\PY{o}{.}\PY{n}{predict}\PY{p}{(}\PY{n}{X\PYZus{}test}\PY{p}{)}
        \PY{n+nb}{print}\PY{p}{(}\PY{l+s+s2}{\PYZdq{}}\PY{l+s+s2}{       actual}\PY{l+s+se}{\PYZbs{}n}\PY{l+s+s2}{       +   \PYZhy{}}\PY{l+s+se}{\PYZbs{}n}\PY{l+s+s2}{pred+[[tp, fp]}\PY{l+s+se}{\PYZbs{}n}\PY{l+s+s2}{dict\PYZhy{} [fn, tn]]}\PY{l+s+s2}{\PYZdq{}}\PY{p}{)}
        \PY{n+nb}{print}\PY{p}{(}\PY{n}{confusion\PYZus{}matrix}\PY{p}{(}\PY{n}{y\PYZus{}test}\PY{p}{,} \PY{n}{y\PYZus{}pred}\PY{p}{)}\PY{p}{,} \PY{l+s+s2}{\PYZdq{}}\PY{l+s+se}{\PYZbs{}n}\PY{l+s+s2}{\PYZdq{}}\PY{p}{)}
        \PY{n+nb}{print}\PY{p}{(}\PY{l+s+s2}{\PYZdq{}}\PY{l+s+se}{\PYZbs{}n}\PY{l+s+s2}{\PYZdq{}}\PY{p}{)}
        \PY{n+nb}{print}\PY{p}{(}\PY{n}{classification\PYZus{}report}\PY{p}{(}\PY{n}{y\PYZus{}test}\PY{p}{,} \PY{n}{y\PYZus{}pred}\PY{p}{)}\PY{p}{)}
\end{Verbatim}


    \begin{Verbatim}[commandchars=\\\{\}]
       actual
       +   -
pred+[[tp, fp]
dict- [fn, tn]]
[[50  4]
 [ 3 86]] 



             precision    recall  f1-score   support

          0       0.94      0.93      0.93        54
          1       0.96      0.97      0.96        89

avg / total       0.95      0.95      0.95       143


    \end{Verbatim}

    This model did pretty good. However we do not want to fail to diagnose
someone who could potentially have breast cancer because they will die
without treatment. False positives are preferred over false negatives
because the patient can get a 2nd opinion.

\subsection{k-folds split and test}\label{k-folds-split-and-test}

What if the data were split differently? Would we get the same result
with similar models?

    \begin{Verbatim}[commandchars=\\\{\}]
{\color{incolor}In [{\color{incolor}8}]:} \PY{n+nb}{print}\PY{p}{(}\PY{n}{y}\PY{o}{.}\PY{n}{shape}\PY{p}{,} \PY{n}{X}\PY{o}{.}\PY{n}{shape}\PY{p}{)}
\end{Verbatim}


    \begin{Verbatim}[commandchars=\\\{\}]
(569,) (569, 30)

    \end{Verbatim}

    \begin{Verbatim}[commandchars=\\\{\}]
{\color{incolor}In [{\color{incolor}24}]:} \PY{n}{k\PYZus{}fold} \PY{o}{=} \PY{n}{KFold}\PY{p}{(}\PY{l+m+mi}{5}\PY{p}{)} \PY{c+c1}{\PYZsh{} 80\PYZpc{} / 20\PYZpc{} train / test splits}
         \PY{n}{split} \PY{o}{=} \PY{n}{k\PYZus{}fold}\PY{o}{.}\PY{n}{split}\PY{p}{(}\PY{n}{X}\PY{p}{,}\PY{n}{y}\PY{p}{)}
         \PY{n}{tt} \PY{o}{=} \PY{p}{[}\PY{p}{]}
         \PY{k}{for} \PY{n}{k}\PY{p}{,} \PY{p}{(}\PY{n}{train}\PY{p}{,}\PY{n}{test}\PY{p}{)} \PY{o+ow}{in} \PY{n+nb}{enumerate}\PY{p}{(}\PY{n}{split}\PY{p}{)}\PY{p}{:}
             \PY{n+nb}{print}\PY{p}{(}\PY{n}{k}\PY{p}{,} \PY{n}{train}\PY{o}{.}\PY{n}{shape}\PY{p}{,} \PY{n}{test}\PY{o}{.}\PY{n}{shape}\PY{p}{)}
             \PY{n}{mlp}\PY{o}{.}\PY{n}{fit}\PY{p}{(}\PY{n}{X}\PY{o}{.}\PY{n}{loc}\PY{p}{[}\PY{n}{train}\PY{p}{]}\PY{p}{,}\PY{n}{y}\PY{p}{[}\PY{n}{train}\PY{p}{]}\PY{p}{)}
             \PY{n}{y\PYZus{}pred} \PY{o}{=} \PY{n}{mlp}\PY{o}{.}\PY{n}{predict}\PY{p}{(}\PY{n}{X}\PY{o}{.}\PY{n}{loc}\PY{p}{[}\PY{n}{test}\PY{p}{]}\PY{p}{)}
             \PY{n}{cm} \PY{o}{=} \PY{n}{confusion\PYZus{}matrix}\PY{p}{(}\PY{n}{y}\PY{p}{[}\PY{n}{test}\PY{p}{]}\PY{p}{,} \PY{n}{y\PYZus{}pred}\PY{p}{)}
             \PY{n}{cr} \PY{o}{=} \PY{n}{classification\PYZus{}report}\PY{p}{(}\PY{n}{y}\PY{p}{[}\PY{n}{test}\PY{p}{]}\PY{p}{,} \PY{n}{y\PYZus{}pred}\PY{p}{)}
             \PY{n}{tt}\PY{o}{.}\PY{n}{append}\PY{p}{(}\PY{p}{[}\PY{n}{train}\PY{p}{,}\PY{n}{test}\PY{p}{,}\PY{n}{cm}\PY{p}{,}\PY{n}{cr}\PY{p}{]}\PY{p}{)}
             \PY{n+nb}{print}\PY{p}{(}\PY{n}{cm}\PY{p}{)}
             \PY{n+nb}{print}\PY{p}{(}\PY{n}{cr}\PY{p}{)}
\end{Verbatim}


    \begin{Verbatim}[commandchars=\\\{\}]
0 (455,) (114,)
[[66  2]
 [ 2 44]]
             precision    recall  f1-score   support

          0       0.97      0.97      0.97        68
          1       0.96      0.96      0.96        46

avg / total       0.96      0.96      0.96       114

1 (455,) (114,)
[[45  4]
 [ 1 64]]
             precision    recall  f1-score   support

          0       0.98      0.92      0.95        49
          1       0.94      0.98      0.96        65

avg / total       0.96      0.96      0.96       114

2 (455,) (114,)
[[38  2]
 [ 0 74]]
             precision    recall  f1-score   support

          0       1.00      0.95      0.97        40
          1       0.97      1.00      0.99        74

avg / total       0.98      0.98      0.98       114

3 (455,) (114,)
[[29  0]
 [ 1 84]]
             precision    recall  f1-score   support

          0       0.97      1.00      0.98        29
          1       1.00      0.99      0.99        85

avg / total       0.99      0.99      0.99       114

4 (456,) (113,)
[[26  0]
 [ 1 86]]
             precision    recall  f1-score   support

          0       0.96      1.00      0.98        26
          1       1.00      0.99      0.99        87

avg / total       0.99      0.99      0.99       113


    \end{Verbatim}

    Several of the five kfolds show a false negative. Iteration '3'
performed the worst.

\subsection{Challenge question}\label{challenge-question}

Invesigate wheter "m-folds validation" variation of "k-folds" be used as
a basis to identify and remove outliers for an improved model.
Prediction: "m-folds" will either be novel idea or about as dumb as a
square wheel.

\begin{itemize}
\tightlist
\item
  "m-folds validation" will

  \begin{itemize}
  \tightlist
  \item
    for every m row:

    \begin{itemize}
    \tightlist
    \item
      train on all the other rows
    \item
      predict the m\_th row with that model
    \item
      keep the m\_th row if prediction was accurate
    \item
      discard the m\_th row if prediction was inaccurate, i.e. remove it
      as an outlier
    \end{itemize}
  \item
    train a model all the kept rows (i.e. without removed outliers)
  \item
    evaluate and compare metrics with previously predicted models
  \end{itemize}
\end{itemize}

    \begin{Verbatim}[commandchars=\\\{\}]
{\color{incolor}In [{\color{incolor}10}]:} \PY{k}{def} \PY{n+nf}{m\PYZus{}folds}\PY{p}{(}\PY{n}{X}\PY{p}{,} \PY{n}{y}\PY{p}{,} \PY{n}{algorithm}\PY{p}{)}\PY{p}{:}
             \PY{n}{keep} \PY{o}{=} \PY{p}{[}\PY{p}{]}
             \PY{n}{discard} \PY{o}{=} \PY{p}{[}\PY{p}{]}
             \PY{k}{for} \PY{n}{i}\PY{p}{,} \PY{n}{\PYZus{}} \PY{o+ow}{in} \PY{n+nb}{enumerate}\PY{p}{(}\PY{n}{y}\PY{p}{)}\PY{p}{:}
                 \PY{n}{train} \PY{o}{=} \PY{n}{np}\PY{o}{.}\PY{n}{arange}\PY{p}{(}\PY{n+nb}{len}\PY{p}{(}\PY{n}{y}\PY{p}{)}\PY{p}{)} \PY{o}{!=} \PY{n}{i}
                 \PY{n}{algorithm}\PY{o}{.}\PY{n}{fit}\PY{p}{(}\PY{n}{X}\PY{p}{[}\PY{n}{train}\PY{p}{]}\PY{p}{,}\PY{n}{y}\PY{p}{[}\PY{n}{train}\PY{p}{]}\PY{p}{)}
                 \PY{n}{y\PYZus{}pred} \PY{o}{=} \PY{n}{algorithm}\PY{o}{.}\PY{n}{predict}\PY{p}{(}\PY{n}{X}\PY{o}{.}\PY{n}{loc}\PY{p}{[}\PY{p}{[}\PY{n}{i}\PY{p}{]}\PY{p}{]}\PY{p}{)}
                 \PY{k}{if} \PY{n}{y\PYZus{}pred}\PY{p}{[}\PY{l+m+mi}{0}\PY{p}{]} \PY{o}{==} \PY{n}{y}\PY{p}{[}\PY{n}{i}\PY{p}{]}\PY{p}{:}
                     \PY{n}{keep}\PY{o}{.}\PY{n}{append}\PY{p}{(}\PY{n}{i}\PY{p}{)}
                 \PY{k}{else}\PY{p}{:}
                     \PY{n}{discard}\PY{o}{.}\PY{n}{append}\PY{p}{(}\PY{n}{i}\PY{p}{)}
             \PY{k}{return} \PY{n}{keep}\PY{p}{,} \PY{n}{discard}
             
\end{Verbatim}


    \begin{Verbatim}[commandchars=\\\{\}]
{\color{incolor}In [{\color{incolor}11}]:} \PY{n}{keep}\PY{p}{,} \PY{n}{discard} \PY{o}{=} \PY{n}{m\PYZus{}folds}\PY{p}{(}\PY{n}{X}\PY{p}{,} \PY{n}{y}\PY{p}{,} \PY{n}{mlp}\PY{p}{)}
         \PY{n+nb}{print}\PY{p}{(}\PY{n+nb}{len}\PY{p}{(}\PY{n}{keep}\PY{p}{)}\PY{p}{,} \PY{n+nb}{len}\PY{p}{(}\PY{n}{discard}\PY{p}{)}\PY{p}{)}
\end{Verbatim}


    \begin{Verbatim}[commandchars=\\\{\}]
C:\textbackslash{}Users\textbackslash{}AFMS\textbackslash{}Anaconda3\textbackslash{}lib\textbackslash{}site-packages\textbackslash{}sklearn\textbackslash{}neural\_network\textbackslash{}multilayer\_perceptron.py:564: ConvergenceWarning: Stochastic Optimizer: Maximum iterations (200) reached and the optimization hasn't converged yet.
  \% self.max\_iter, ConvergenceWarning)

    \end{Verbatim}

    \begin{Verbatim}[commandchars=\\\{\}]
552 17

    \end{Verbatim}

    \begin{Verbatim}[commandchars=\\\{\}]
{\color{incolor}In [{\color{incolor}13}]:} \PY{n}{y}\PY{p}{[}\PY{n}{discard}\PY{p}{]}
\end{Verbatim}


\begin{Verbatim}[commandchars=\\\{\}]
{\color{outcolor}Out[{\color{outcolor}13}]:} 40     0
         73     0
         81     1
         99     0
         135    0
         146    0
         157    1
         190    0
         197    0
         213    0
         215    0
         255    0
         291    1
         297    0
         363    1
         526    1
         541    1
         Name: y, dtype: int32
\end{Verbatim}
            
    Create a model based on all the kept rows and test with discarded
rows.\\
Expect this model to get every answer wrong as it is trained with less
data than all the models created with "m-folds".

    \begin{Verbatim}[commandchars=\\\{\}]
{\color{incolor}In [{\color{incolor}15}]:} \PY{n}{mlp}\PY{o}{.}\PY{n}{fit}\PY{p}{(}\PY{n}{X}\PY{o}{.}\PY{n}{loc}\PY{p}{[}\PY{n}{keep}\PY{p}{]}\PY{p}{,} \PY{n}{y}\PY{p}{[}\PY{n}{keep}\PY{p}{]}\PY{p}{)}
         \PY{n}{y\PYZus{}pred}\PY{o}{=} \PY{n}{mlp}\PY{o}{.}\PY{n}{predict}\PY{p}{(}\PY{n}{X}\PY{o}{.}\PY{n}{loc}\PY{p}{[}\PY{n}{discard}\PY{p}{]}\PY{p}{)}
         \PY{n+nb}{print}\PY{p}{(}\PY{n}{confusion\PYZus{}matrix}\PY{p}{(}\PY{n}{y}\PY{p}{[}\PY{n}{discard}\PY{p}{]}\PY{p}{,} \PY{n}{y\PYZus{}pred}\PY{p}{)}\PY{p}{)}
         \PY{n+nb}{print}\PY{p}{(}\PY{n}{classification\PYZus{}report}\PY{p}{(}\PY{n}{y}\PY{p}{[}\PY{n}{discard}\PY{p}{]}\PY{p}{,} \PY{n}{y\PYZus{}pred}\PY{p}{)}\PY{p}{)}
\end{Verbatim}


    \begin{Verbatim}[commandchars=\\\{\}]
[[3 8]
 [3 3]]
             precision    recall  f1-score   support

          0       0.50      0.27      0.35        11
          1       0.27      0.50      0.35         6

avg / total       0.42      0.35      0.35        17


    \end{Verbatim}

    Wow. This model was expected to get 0\% correct and it did bettter.\\
Perhaps "m-folds" does something.\\
Let's exclude the m\_folds discarded record from the worst k\_means
iteration model and see if it does any better.

    \begin{Verbatim}[commandchars=\\\{\}]
{\color{incolor}In [{\color{incolor}43}]:} \PY{n}{train} \PY{o}{=} \PY{n}{tt}\PY{p}{[}\PY{l+m+mi}{1}\PY{p}{]}\PY{p}{[}\PY{l+m+mi}{0}\PY{p}{]}
         \PY{n}{test} \PY{o}{=} \PY{n}{tt}\PY{p}{[}\PY{l+m+mi}{1}\PY{p}{]}\PY{p}{[}\PY{l+m+mi}{1}\PY{p}{]}
         \PY{n+nb}{len}\PY{p}{(}\PY{n}{train}\PY{p}{)}
\end{Verbatim}


\begin{Verbatim}[commandchars=\\\{\}]
{\color{outcolor}Out[{\color{outcolor}43}]:} 455
\end{Verbatim}
            
    \begin{Verbatim}[commandchars=\\\{\}]
{\color{incolor}In [{\color{incolor}44}]:} \PY{n}{mask} \PY{o}{=} \PY{n}{np}\PY{o}{.}\PY{n}{isin}\PY{p}{(}\PY{n}{train}\PY{p}{,}\PY{n}{keep}\PY{p}{)}
         \PY{n}{tr} \PY{o}{=} \PY{n}{train}\PY{p}{[}\PY{n}{mask}\PY{p}{]}
         \PY{n+nb}{len}\PY{p}{(}\PY{n}{tr}\PY{p}{)}
\end{Verbatim}


\begin{Verbatim}[commandchars=\\\{\}]
{\color{outcolor}Out[{\color{outcolor}44}]:} 445
\end{Verbatim}
            
    \begin{Verbatim}[commandchars=\\\{\}]
{\color{incolor}In [{\color{incolor}45}]:} \PY{n}{mlp}\PY{o}{.}\PY{n}{fit}\PY{p}{(}\PY{n}{X}\PY{o}{.}\PY{n}{loc}\PY{p}{[}\PY{n}{tr}\PY{p}{]}\PY{p}{,}\PY{n}{y}\PY{p}{[}\PY{n}{tr}\PY{p}{]}\PY{p}{)}
         \PY{n}{y\PYZus{}pred} \PY{o}{=} \PY{n}{mlp}\PY{o}{.}\PY{n}{predict}\PY{p}{(}\PY{n}{X}\PY{o}{.}\PY{n}{loc}\PY{p}{[}\PY{n}{test}\PY{p}{]}\PY{p}{)}
         \PY{n+nb}{print}\PY{p}{(}\PY{n}{confusion\PYZus{}matrix}\PY{p}{(}\PY{n}{y}\PY{p}{[}\PY{n}{test}\PY{p}{]}\PY{p}{,} \PY{n}{y\PYZus{}pred}\PY{p}{)}\PY{p}{)}
         \PY{n+nb}{print}\PY{p}{(}\PY{n}{classification\PYZus{}report}\PY{p}{(}\PY{n}{y}\PY{p}{[}\PY{n}{test}\PY{p}{]}\PY{p}{,} \PY{n}{y\PYZus{}pred}\PY{p}{)}\PY{p}{)}
\end{Verbatim}


    \begin{Verbatim}[commandchars=\\\{\}]
[[44  5]
 [ 1 64]]
             precision    recall  f1-score   support

          0       0.98      0.90      0.94        49
          1       0.93      0.98      0.96        65

avg / total       0.95      0.95      0.95       114


    \end{Verbatim}

    \begin{Verbatim}[commandchars=\\\{\}]
{\color{incolor}In [{\color{incolor}47}]:} \PY{n+nb}{print}\PY{p}{(}\PY{n}{tt}\PY{p}{[}\PY{l+m+mi}{1}\PY{p}{]}\PY{p}{[}\PY{l+m+mi}{2}\PY{p}{]}\PY{p}{)}
         \PY{n+nb}{print}\PY{p}{(}\PY{n}{tt}\PY{p}{[}\PY{l+m+mi}{1}\PY{p}{]}\PY{p}{[}\PY{l+m+mi}{3}\PY{p}{]}\PY{p}{)}
\end{Verbatim}


    \begin{Verbatim}[commandchars=\\\{\}]
[[45  4]
 [ 1 64]]
             precision    recall  f1-score   support

          0       0.98      0.92      0.95        49
          1       0.94      0.98      0.96        65

avg / total       0.96      0.96      0.96       114


    \end{Verbatim}

    Not much improvement. We got one more true positive and the same false
negatives. The f1-score was a little better. Let's comparare discarded
indices to those records which we might otherwise condsider outliers.

    \begin{Verbatim}[commandchars=\\\{\}]
{\color{incolor}In [{\color{incolor}48}]:} \PY{n}{X}\PY{o}{.}\PY{n}{boxplot}\PY{p}{(}\PY{p}{)}
\end{Verbatim}


\begin{Verbatim}[commandchars=\\\{\}]
{\color{outcolor}Out[{\color{outcolor}48}]:} <matplotlib.axes.\_subplots.AxesSubplot at 0x25906ceccc0>
\end{Verbatim}
            
    \begin{center}
    \adjustimage{max size={0.9\linewidth}{0.9\paperheight}}{output_27_1.png}
    \end{center}
    { \hspace*{\fill} \\}
    
    Looks like there are a lot of outliers. The most extreme ones have
values of \textgreater{} 6. Let's get the indicies of these values.

    \begin{Verbatim}[commandchars=\\\{\}]
{\color{incolor}In [{\color{incolor}64}]:} \PY{n}{np}\PY{o}{.}\PY{n}{isin}\PY{p}{(}\PY{n}{discard}\PY{p}{,} \PY{n}{X}\PY{p}{[}\PY{n}{X}\PY{o}{\PYZgt{}}\PY{l+m+mi}{6}\PY{p}{]}\PY{o}{.}\PY{n}{stack}\PY{p}{(}\PY{p}{)}\PY{o}{.}\PY{n}{reset\PYZus{}index}\PY{p}{(}\PY{p}{)}\PY{p}{[}\PY{l+s+s1}{\PYZsq{}}\PY{l+s+s1}{level\PYZus{}0}\PY{l+s+s1}{\PYZsq{}}\PY{p}{]}\PY{p}{)}
\end{Verbatim}


\begin{Verbatim}[commandchars=\\\{\}]
{\color{outcolor}Out[{\color{outcolor}64}]:} array([False, False, False, False, False, False, False,  True, False,
                 True, False, False, False, False, False, False, False])
\end{Verbatim}
            
    Meh. Only two extreme outliers were shared. Looks like m\_means is not
so great at finding outliers.


    % Add a bibliography block to the postdoc
    
    
    
    \end{document}
